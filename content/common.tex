\part{Структура символьной последовательности}

Обозначим символьную последовательность длинной $l$ состоящей из $m$ различных символов как $S^{l}_{m}$.
Для исследования стуктуры символьной последовательности рассмотрим данный как совокупность "строя" символьной последовательности
и алфавита на котором определена данная символьная последовательность $S^{l}_{m}=(A_{m}, O^{l}_{m})$, 
где $A_{m}$ - алфавит мощностью $m$, элементы которого отсортированы в порядке их появления в исходной последовательности $S^{l}_{m}$, 
а $O^{l}_{m}$ - строй цепи.
Определение строя символьной последовательности и процедура его получения  приведена в работе Гуменюка А.С. \cite{gumenuk_base}.
Опишим процедуру выделения строя цепи следующием, отличающиемся от оригинальной рядом шагов, но дающеё на выходе эквивалентный объект.

\subparagraph{Процедура выделения строя}

\begin{enumerate}
 \item Выделите алфавит $A_{m}$ как множество уникальных символов, входящих в исходную последовательность, 
       с сохранением порядка их появления  в исходной последовательности.
 \item Выделите строй цепи $O^{l}_{m}$ заменяя каждый символ исходной последовательности его порядковым номером в $A_{m}$
\end{enumerate}

\subparagraph{Пример}

Дано:\\
  Исходная символьная последовательность $S^{10}_{4}$\\ 
  $S^{10}_{4} = \begin{tabular}{|c|c|c|c|c|c|c|c|c|c|}
  \hline
  T & C & A & C & A & T & C & A & G & C \\
  \hline
  \end{tabular}$
 
 Алфавит данной последовательности будет $A_{m} = (T, C, A, G)$.
 Eсли рассмотреть $A_{m}$ как множество пар (порядковый номер, символ), то алфавит принимает вид $A_{m} = ((1, T), (2, C), (3, A), (4, G))$.\\
 Получим строй заменив символы последовательности на их порядковые номера в алфавите. \\
  $O^{10}_{4} = \begin{tabular}{|c|c|c|c|c|c|c|c|c|c|}
  \hline
  1 & 2 & 3 & 2 & 3 & 1 & 2 & 3 & 4 & 2 \\
  \hline
  \end{tabular}$ \\
 
 Количество различных строев длиной $l$ с мощностью алфавита $m$ явлется числом Стирлинга второго рода, что показано в работе 
 \cite{semenov_magistr}.\\

 \begin{table}[all_orders]
  \centering
  \caption{Все возможные различные строи в зависимости от мощности алвфавита и длины цепи}
  \label{my-label}
  \begin{tabular}{@{}|l|l|lll@{}}
     \hline
     l /\ m                                   & 1                                         & \multicolumn{1}{l|}{2}   & \multicolumn{1}{l|}{3}                    & \multicolumn{1}{l|}{4}                     \\ \midrule
     \multicolumn{1}{|c|}{1}                  & \multicolumn{1}{c|}{1}                    & \multicolumn{1}{c}{}     & \multicolumn{1}{c}{}                      & \multicolumn{1}{c}{}                       \\ \cmidrule(r){1-3}
     \multicolumn{1}{|c|}{2}                  & \multicolumn{1}{c|}{11}                   & \multicolumn{1}{c|}{12}  & \multicolumn{1}{c}{}                      & \multicolumn{1}{c}{}                       \\ \cmidrule(r){1-4}
     \multicolumn{1}{|c|}{\multirow{3}{*}{3}} & \multicolumn{1}{c|}{\multirow{3}{*}{111}} & \multicolumn{1}{c|}{112} & \multicolumn{1}{c|}{\multirow{3}{*}{123}} & \multicolumn{1}{c}{\multirow{3}{*}{}}      \\
     \multicolumn{1}{|c|}{}                   & \multicolumn{1}{c|}{}                     & \multicolumn{1}{c|}{121} & \multicolumn{1}{c|}{}                     & \multicolumn{1}{c}{}                       \\
     \multicolumn{1}{|c|}{}                   & \multicolumn{1}{c|}{}                     & \multicolumn{1}{c|}{122} & \multicolumn{1}{c|}{}                     & \multicolumn{1}{c}{}                       \\ \cmidrule(r){1-5} 
     \multirow{7}{*}{4}                       & \multirow{7}{*}{1111}                     & 1112                     & \multicolumn{1}{|l|}{1123}                 & \multicolumn{1}{l|}{\multirow{7}{*}{1234}} \\
                                              &                                           & 1121                     & \multicolumn{1}{|l|}{1213}                 & \multicolumn{1}{l|}{}                      \\
                                              &                                           & 1122                     & \multicolumn{1}{|l|}{1223}                 & \multicolumn{1}{l|}{}                      \\
                                              &                                           & 1211                     & \multicolumn{1}{|l|}{1231}                 & \multicolumn{1}{l|}{}                      \\
                                              &                                           & 1212                     & \multicolumn{1}{|l|}{1232}                 & \multicolumn{1}{l|}{}                      \\
                                              &                                           & 1221                     & \multicolumn{1}{|l|}{1233}                 & \multicolumn{1}{l|}{}                      \\
                                              &                                           & 1222                     & \multicolumn{1}{|l|}{}                     & \multicolumn{1}{l|}{}                      \\ \bottomrule
   \end{tabular}
\end{table}



 \begin{table}[all_orders_count]
\centering
\caption{Количество всевозможных строев в зависимости от мощности алфавита и длины последовательности}
\label{}
\begin{tabular}{@{}|c|c|ccc@{}}
\toprule
\multicolumn{1}{|l|}{l /\ m} & \multicolumn{1}{l|}{1} & \multicolumn{1}{l|}{2} & \multicolumn{1}{l|}{3} & \multicolumn{1}{l|}{4} \\ \midrule
1                         & 1                      &                        &                        &                        \\ \cmidrule(r){1-3}
2                         & 1                      & \multicolumn{1}{c|}{1} &                        &                        \\ \cmidrule(r){1-4}
3                         & 1                      & \multicolumn{1}{c|}{3} & \multicolumn{1}{c|}{1} &                        \\ \midrule
4                         & 1                      & \multicolumn{1}{c|}{7} & \multicolumn{1}{c|}{6} & \multicolumn{1}{c|}{1} \\ \bottomrule
\end{tabular}
\end{table}


 Количество строк с одинаковым строем будет равно количеству перестановок элементов алфавита $m!$. \\

 \begin{table}[all_strings_count]
\centering
\caption{Количество всевозможных строк в зависимости от мощности алфавита и длины последовательности}
\label{}
\begin{tabular}{@{}|c|c|ccc@{}}
\toprule
\multicolumn{1}{|l|}{l /\ m} & \multicolumn{1}{l|}{1} & \multicolumn{1}{l|}{2} & \multicolumn{1}{l|}{3} & \multicolumn{1}{l|}{4} \\ \midrule
1                         & 1                      &                        &                        &                        \\ \cmidrule(r){1-3}
2                         & 1                      & \multicolumn{1}{c|}{2} &                        &                        \\ \cmidrule(r){1-4}
3                         & 1                      & \multicolumn{1}{c|}{6} & \multicolumn{1}{c|}{6} &                        \\ \midrule
4                         & 1                      & \multicolumn{1}{c|}{14} & \multicolumn{1}{c|}{36} & \multicolumn{1}{c|}{24} \\ \bottomrule
\end{tabular}
\end{table}





