\part{Структура символьной последовательности}

Обозначим символьную последовательность длинной $l$ состоящей из $m$ различных символов как $S^{l}_{m}$.
Для исследования стуктуры символьной последовательности рассмотрим данный как совокупность "строя" символьной последовательности
и алфавита на котором определена данная символьная последовательность $S^{l}_{m}=(A_{m}, O^{l}_{m})$, 
где $A_{m}$ - алфавит мощностью $m$, элементы которого отсортированы в порядке их появления в исходной последовательности $S^{l}_{m}$, 
а $O^{l}_{m}$ - строй цепи.
Определение строя символьной последовательности и процедура его получения  приведена в работе Гуменюка А.С. \cite{gumenuk_base}.
Опишим процедуру выделения строя цепи следующием, отличающиемся от оригинальной рядом шагов, но дающеё на выходе эквивалентный объект.

\subparagraph{Процедура выделения строя}

\begin{enumerate}
 \item Выделите алфавит $A_{m}$ как множество уникальных символов, входящих в исходную последовательность, 
       с сохранением порядка их появления  в исходной последовательности.
 \item Выделите строй цепи $O^{l}_{m}$ заменяя каждый символ исходной последовательности его порядковым номером в $A_{m}$
\end{enumerate}

\subparagraph{Пример}

Дано:\\
  Исходная символьная последовательность $S^{10}_{4}$\\ 
  $S^{10}_{4} = \begin{tabular}{|c|c|c|c|c|c|c|c|c|c|}
  \hline
  T & C & A & C & A & T & C & A & G & C \\
  \hline
  \end{tabular}$
 
 Алфавит для данной последовательности будет $A_{m} = (T, C, A, G)$.
 Eсли рассмотреть алфавит как множество пар (порядковый номер, символ), то алфавит принимает вид $A_{m} = ((1, T), (2, C), (3, A), (4, G))$.\\
 Тогда получим строй заменив символы последовательности на их порядковые номера в алфавите. \\
  $O^{10}_{4} = \begin{tabular}{|c|c|c|c|c|c|c|c|c|c|}
  \hline
  1 & 2 & 3 & 2 & 3 & 1 & 2 & 3 & 4 & 2 \\
  \hline
  \end{tabular}$
 

